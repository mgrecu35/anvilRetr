%% Version 6.1, 1 September 2021
%
%%%%%%%%%%%%%%%%%%%%%%%%%%%%%%%%%%%%%%%%%%%%%%%%%%%%%%%%%%%%%%%%%%%%%%
% TemplateV6.1.tex --  LaTeX-based blank template for submissions to the 
% American Meteorological Society
%
%%%%%%%%%%%%%%%%%%%%%%%%%%%%%%%%%%%%%%%%%%%%%%%%%%%%%%%%%%%%%%%%%%%%%
% PREAMBLE
%%%%%%%%%%%%%%%%%%%%%%%%%%%%%%%%%%%%%%%%%%%%%%%%%%%%%%%%%%%%%%%%%%%%%

%% Start with one of the following:
% 1.5-SPACED VERSION FOR SUBMISSION TO THE AMS
\documentclass{ametsocV6.1}

% TWO-COLUMN JOURNAL PAGE LAYOUT---FOR AUTHOR USE ONLY
% \documentclass[twocol]{ametsocV6.1}

%%%%%%%%%%%%%%%%%%%%%%%%%%%%%%%%

%%% To be entered by author:

%% May use \\ to break lines in title:

\title{Synergistic retrievals of ice in high clouds from lidar, 
Ku-band radar and submillimeter wave radiometer observations}

%% Enter authors' names and affiliations as you see in the examples below.
%
%% Use \correspondingauthor{} and \thanks{} (\thanks command to be used for affiliations footnotes, 
%% such as current affiliation, additional affiliation, deceased, co-first authors, etc.)
%% immediately following the appropriate author.
%
%% Note that the \correspondingauthor{} command is NECESSARY.
%% The \thanks{} commands are OPTIONAL.
%
%% Enter affiliations within the \affiliation{} field. Use \aff{#} to indicate the affiliation letter at both the
%% affiliation and at each author's name. Use \\ to insert line breaks to place each affiliation on its own line.

%\authors{Author One,\aff{a}\correspondingauthor{Author One, email@email.com} 
%Author Two,\aff{a} 
%Author Three,\aff{b} 
%Author Four,\aff{a} 
%Author Five\thanks{Author Five's current affiliation: NCAR, Boulder, Colorado},\aff{c} 
%Author Six,\aff{c} 
%Author Seven,\aff{d}
% and Author Eight\aff{a,d}
%}
%
%\affiliation{\aff{a}{First Affiliation}\\
%\aff{b}{Second Affiliation}\\
%\aff{c}{Third Affiliation}\\
%\aff{d}{Fourth Affiliation}
%}

\authors{Mircea Grecu\aff{a}\correspondingauthor{Mircea Grecu, mircea.grecu-1@nasa.gov}
and John Yorks \aff{a}}
\affiliation{\aff{a}{NASA GSFC}}

%%%%%%%%%%%%%%%%%%%%%%%%%%%%%%%%%%%%%%%%%%%%%%%%%%%%%%%%%%%%%%%%%%%%%
% ABSTRACT
%
% Enter your abstract here
% Abstracts should not exceed 250 words in length!
%
 
\abstract{Enter the text of your abstract here.}

\begin{document}

%% Necessary!
\maketitle

%%%%%%%%%%%%%%%%%%%%%%%%%%%%%%%%%%%%%%%%%%%%%%%%%%%%%%%%%%%%%%%%%%%%%
% SIGNIFICANCE STATEMENT/CAPSULE SUMMARY
%%%%%%%%%%%%%%%%%%%%%%%%%%%%%%%%%%%%%%%%%%%%%%%%%%%%%%%%%%%%%%%%%%%%%
%
% If you are including an optional significance statement for a journal article or a required capsule summary for BAMS 
% (see www.ametsoc.org/ams/index.cfm/publications/authors/journal-and-bams-authors/formatting-and-manuscript-components for details), 
% please apply the necessary command as shown below:
%
% Significance Statement (all journals except BAMS)
%

%\statement
%This paper is significant.
%	 Enter significance statement here, no more than 120 words. See \url{www.ametsoc.org/index.cfm/ams/publications/author-information/significance-statements/} for details.
%
%% Capsule (BAMS only)
%%
%\capsule
%       Enter BAMS capsule here, no more than 30 words. See \url{www.ametsoc.org/index.cfm/ams/publications/author-information/formatting-and-manuscript-components/#capsule} for details.
%
%% * * If using twocol mode, you will need to use the commands "twocolsig" and "twocolcapsule" in place of "sig" and "capsule"
%%      to ensure that the text box correctly spans across both columns.
%

%%%%%%%%%%%%%%%%%%%%%%%%%%%%%%%%%%%%%%%%%%%%%%%%%%%%%%%%%%%%%%%%%%%%%
% MAIN BODY OF PAPER
%%%%%%%%%%%%%%%%%%%%%%%%%%%%%%%%%%%%%%%%%%%%%%%%%%%%%%%%%%%%%%%%%%%%%
%

%% In all cases, if there is only one entry of this type within
%% the higher level heading, use the star form: 
%%
\section{Introduction}
The future NASA Atmospheric Observing System (AOS) mission (Braun 2022) is expected to feature new 
combinations of observations that may be used to quantify the amounts of ice in high clouds and 
characterize the microphysical properties of ice particles. These observations include lidar backscatter,
 Ku-band radar reflectivity and submillimeter wave radiometer brightness temperature measurements.  
 While not optimal for cloud ice estimation, but for the characterization of a broader spectrum of 
 cloud and precipitation processes, these observations are nevertheless synergistic from the 
 characterization of ice clouds perspective. That is, despite the fact that lidar observations 
 attenuate quickly in thick ice clouds and the Ku-band radar will not be able to detect clouds 
 characterized by an echo weaker than 8.0 dBZ, the active observations are expected to provide context 
 that may be incorporated into radiometer retrievals. In this study, we investigate the impact of 
 incorporating the lidar and radar observations into the radiometer retrieval of ice clouds.
 Because the existent amount of coincident backscatter lidar, Ku-band radar, and submillimeter-wave 
 radiometer observations is rather insufficient to derive conclusive results, we employ accurate 
 physical models to simulate the lidar, radar and radiometer observations and use a cross-validation 
 methodology to characterize the retrieval accuracy. As retrievals from passive instrument 
 observations strongly depend on the "a priori" information (Rodgers 2000), for the results to be 
 relevant in real applications it is necessary to base them on realistic vertical distributions of 
 ice properties.  Such distributions may be derived from cloud-resolving-model (CRM) simulations 
 (Pfreundschuh et al. 2020) or directly from observations.  In this study, we employ the latter 
 approach, as CRMs may still be deficient in properly reproducing the vertical distribution of ice 
 clouds and their associated microphysical properties.  Specifically, we use observation and
 products from the CloudSat (CS) mission (Stephens et al. 2002) to derive a database of ice microphysical
 properties and associated simulated lidar, radar and radiometer observations.  The database is used to
investigate the accuracy of the retrieved ice cloud properties from the simulated observations. The article
is organized as follows.  In Section 2, we describe the approach used to derive the ice properties and
the associated simulated observations, the retrieval and the evaluation methodology.  In Section 3, we 
present the results of the evaluation methodology. We conclude in Section 4.

% \subsection*{subsection}
% text...
\section{Methodology}
As previously mentioned, we use CloudSat (CS) observations (Stephens et 2002) to derive the vertical 
distributions of ice properties needed in the investigation.  Although research quality CS cloud ice 
products exist, to maximize the physical consistency of the approach, we do not use them but derive ice amounts
and associated properties directly from CS reflectivity observations.  This ensures the consistency between 
the particle
distribution assumptions and the electromagnetic scattering properties used in the CS reflectivity processing 
and those used the simulation of the lidar, Ku-band radar and radiometer observations.  Lidar, Ku-band radar 
and submillimeter-wave radiometer observations are simulated from CS observations using accurate physical 
models and realistic assumptions consistent with the most recent knowledge in the field of ice cloud 
microphysics, and a non-parametric estimation methodology based on the k-Means clustering algorithm 
\cite{mackay2003information} is used to 
investigate the instrument synergy.  Details of the methodology are presented below.

\subsection{Assumptions and forward models}
To quantify the number of ice particles in an elementary atmospheric volume as a function of their size, we use
normalized gamma functions (Bringi et al. 2003).  The benefit of normalized gamma functions is that they 
encapsulate the variability of Ice Water Content (IWC) - reflectivity relationship into a single parameter,
i.e. the 
normalized Particle Size Distribution (PSD) intercept (Testud et al. 2001; Bringi et al. 2003). The normalized
PSD intercept is defined as $N_w=\frac {4^4} {\pi \rho_w} \frac {IWC} {D_m^4}$, where $IWC$ is the ice water 
content associated with the PSD, and $D_m$ is the mass weighted mean diameter.  Testud et al. (2001) showed 
that the variability in IWC reflectivity (Z) relationships may be fully explained by variability in $N_w$, and 
that a formula of the type
\begin{equation}
IWC=N_w^{1-b}aZ^b \label{eq:1}
\end{equation}
perfectly explains the relationships between IWC and Z calculated from observed PSDs. Equation (\ref{eq:1}) is not 
sufficient to derive accurate, unbiased estimates of ice water contents, because $N_w$ varies considerably in 
time and space. Nevertheless, multiple studies showed that it is beneficial to parameterize $N_w$ as a function 
of various variables, such as temperature (e.g. Hogan et al. 2006; Delanoe and Hogan 2008; Deng et al. 2010), 
rather than using $N_w$ independent relations.  In this study, we parameterize
$N_w$ as a function of temperature based on the CloudSat 2C-ICE product (Deng et al. 2010; Deng et al. 2013).  
Specifically, we cluster, based on similarity, a large set 2C-ICE profiles into 18 classes using a k-Means 
procedure. The mean IWC profiles associated with the 18 classes are shown in continuous lines in Figure 1.  
Alternative estimates, derived using PSD assumptions and electromagnetic scattering calculations that enable 
accurate and physically consistent simulations of radar observations at Ku-band and radiometer observations of 
submillimeter-wave frequencies are also shown in Figure 1. These estimates are based on the self-similarity 
Rayleigh-Gans approximation (SSRGA) of Hogan et al. (2017). Details regarding the
estimation process are provided in the following paragraphs.  As apparent in Figure 1, the CS and SSRGA estimates are in good 
agreement.  Some discrepancies due to differences between the SSRGA $N_w$ parameterization and the CS 2C-ICE 
"a priori assumptions" are also apparent, but they are not deemed critical in this study, whose objective is 
the investigation of synergistic lidar, Ku-band radar and submillimeter-wave radiometer retrievals, because the 
outcome is not likely to be sensitive to such details.
\begin{figure}[t]
    \centering
    \includegraphics[width=0.75\textwidth,angle=0]{./Figs/fig01.png}\\
    \caption{Mean CS IWC profiles for 18 classes derived using the k-Means clustering algorithm. Associated 
    mean profiles derived from CS reflectivity observations derived using SSRGA scattering calculations 
    and $N_w$ parameterization developed in this study are shown using symbol *. The vertical coordinate 
    is defined relative to the freezing level}\label{f1}
\end{figure}
One may notice that the average IWC profiles in Figure 1 are characterized by different peak values and heights. 
This facilitates a simple way to reverse-engineer to (some extent) the "a priori" assumptions used in the CS 
2C-ICE product and use them in formulation of the type described in Equation (1).  Specifically, the  
derivation of relationships of the type $IWC=a_i Z^{b_i}$ for every class i may be used to study $a_i$ as a
function of height. 
Shown in Figure 2 is a representation of the class multiplicative coefficient $a_i$ as a function of
relative height scatter plot.  As 
apparent in the figure, and as expected, $a_i$ exhibits a strong variation with the relative height. 
Coefficient $b_i$ exhibits a height dependency as well (not shown), but the range of variation is significantly
smaller, 
almost zero relative to the mean value of $b$. Given that any deviation of the multiplicative coefficient in 
an IWC-Z relation from an average is equivalent to a deviation of the associated $N_w$ from its mean value 
(Testud et al. 2001), the variation of $a$ as a function of relative-height may be converted into a $N_w$ 
as a function of relative-height relationship.  We, therefore, use the data in Fig. 2 to parameterize $N_w$ as a 
function of the relative height.

\begin{figure}[t]
    \centering
    \includegraphics[width=0.75\textwidth,angle=0]{./Figs/fig02.png}\\
    \caption{}\label{f2}
\end{figure}

For the determination of reference $a$ and $b$ values to be used with Equation (\ref{eq:1}), we assume that 
PSDs are normalized gamma  distributions with $N_w=0.08cm^{-4}$ and $\mu=2$ and calculate 

\begin{equation}
 Z=\frac {\lambda ^4} {\pi ^5 |K_w|^2} \int_0^{\infty} N(D,D_m) \sigma _b(D) dD
\end{equation}

where $\lambda$ is the radar frequency, $|K_w|$ is the dielectric factor of water, $N(D,D_m) dD$ is the number 
of ice particles of diameter with D and D+dD per unit volume, $D_m$ is the mass weighted mean diameter of the 
distribution, and $\sigma _b(D)$ is the backscattering cross-section of ice particle of diameter $D$. The mass 
weighted mean diameter is equidistantly sampled to span the entire range of IWC values in the CS 2C-ICE dataset.
The assumed mass-size relation is that of Brown and Francis (1995) because it works well with the 
SSRGA scattering calculations (Heymsfield et al. 2022).  The open source software scatter-1.1 of Hogan (2019) 
is used to provide the actual scattering properties. The SSRGA theory was developed for millimeter and 
submillimeter-wave calculations and may not be applicable at lidar's wavelength.  Therefore, for lidar 
calculations, we use the Mie solution included in the scatter-1.1 package. Although more accurate calculations 
based on more 
realistic ice particle shapes exist, they are rather incomplete and not readily available.  Moreover, Wagner 
and Deleny (2022) compared lidar backscatter observations with backscatter calculations based on coincident PSD 
observations and the Mie solution and found good agreement, which suggests that electromagnetic properties 
derived from Mie calculations are adequate for practical applications. The lidar molecular backscatter and 
extinction are calculated using the lidar module of the CFMIP Observation Simulator Package 
(COSP; Bodas-Salcedo et al. 2011).  To account for multiple-scattering in the lidar observations, we are using 
the multiscatter-1.2.11 model (Hogan 2015) of Hogan and Battaglia (2008).  Shown in Figure 3 are the
distributions of simulated Ku-band radar reflectivity and lidar bacscatter as function of height above 
the freezing level.

\begin{figure}[t]
    \centering
    \includegraphics[width=0.75\textwidth,angle=0]{./Figs/fig03.png}\\
    \caption{Simulated distributions of Ku-band radar reflectivity (left) and lidar bacscatter (right)
    as function of height above 
    the freezing level}\label{f3}
\end{figure}

The radiometer observations are calculated using a one-dimensional efficient, but accurate, radiative transfer 
solver based on Eddington's approximation (Kummerow 1993). The Eddington's approximation has been found to work 
well in cloud and precipitation retrieval application despite its simplicity relative to more general (but also 
computationally intensive) approaches such as the Monte Carlo radiative transfer solvers (Liu et al. 1996). 
It should be noted tough that the phase functions of ice particles tend to be highly asymmetric at 
sub-millimeter wave frequencies. For radiative transfer solutions based on the Eddington's approximation to be 
accurate it is necessary that the delta-scaling approach (Joseph et al. 1976) be employed. The delta-scaling 
approach transforms the initial radiative transfer equation into an equivalent one characterized by a less 
asymmetric scattering function and more extinction, which makes the solution Eddington approximation more stable 
and accurate. The absorption due to water vapor and other gases is quantified using the Rosenkranz model 
(Rosenkranz 1998).  The water vapor, temperature and pressure distributions are derived based on a WRF 
simulation of summer convection over the United States.  Specifically, the water vapor, temperature and 
pressure profiles associated with times and areas where the model produces anvils are selected and clustered 
into 40 classes using the k-Means approach.  The mean extinction profiles at the radiometer frequencies are 
calculated for every class 
and used in process of calculating the brightness temperatures from the estimated ice profiles using a simple 
Monte Carlo procedure.  That is, given a retrieved ice profile and its scattering property, an anvil class 
and its associated absorption, temperature and pressure profiles are randomly selected and attached to the 
ice scattering properties. To make the procedure physically meaningful, temperature rather than height is 
used in the ice scattering-gas absorption collocation process.  The emissivities are randomly chosen between 
0.8 and 1.0 and assumed constant for all radiometer frequencies.  Brightness temperatures are calculated at 
89-, 183.31 ± 1.1, and 325.15 ± 1.5 GHz, which correspond to three of the 10 channels of the SAPHIR-NG 
radiometer envisioned to be deployed in the AOS mission (Brogniez et al. 2022).  The other channels are 
centered on the same water vapor absorption lines and are not likely to offer additional information in 
this rather controlled experiment. Nevertheless, the other channels are expected to be useful in reducing the 
uncertainties caused by variability in the vertical distribution of water vapor, which may be greater in 
real life than in the simulated environment. 

The processing steps used to process the CS reflectivity observations and calculate the lidar, Ku-band and 
submillimeter-wave radiometer observations may be summarized as follows:
\begin{enumerate}
\item Derivation of physically consistent radar and radiometer lookup tables to relate basic radar and 
radiometer properties (e.g. reflectivity, attenuation, extinction, scattering-albedo, etc.) to PSD parameters 
such as $IWC$ and $D_m$. The tables are derived for a single of $N_w$, but are usable with any value of 
$N_w$ using the "normalization" operations described in Grecu et al. (2011).
\item Derivation of Nw-relative height parameterization using the 2C-ICE product.
\item Estimation of IWC and related PSD parameters from CS W-band radar observations, using the tables 
constructed
in Step 1 and parameterization derived in Step 2.
\item Calculation of lidar, Ku-band radar and radiometer observations from the estimates derived in Step 3 and 
the tables obtained in Step 1.
\end{enumerate}

The application of these steps produces a large dataset of approximately 200,000 cloud ice profiles and 
associated lidar, radar and radiometer observations that may be used to investigate the synergy of the three 
sensors. Details are provided in the next section.

\subsection{Estimation and evaluation}
Given that the lidar observations may attenuate quickly in thick clouds, while the Ku-band radar will not detect
clouds with an echo weaker than 8.0 dBZ, the radiometer is the instrument likely to provide by itself the most 
complete information about the total amount of ice in its observing volume.  However, the vertical distribution 
of ice is difficult to quantify from radiometer-only observations, because significantly different ice vertical 
distributions may lead to very similar radiometer observations.  This makes radiometer-only retrievals highly 
dependent on the "a priori" information on the distribution of ice clouds in the atmosphere. As previously 
mentioned, this is the reason why CS-based IWC retrievals were preferred to CRM simulations, as retrievals are 
expected to exhibit more natural and less biased distributions. 

We employ a two-step estimation methodology similar to that of \cite{grecu2018}.  In the first step, we 
estimate from the simulated observations the IWC class, out of the 18 classes of shown in Figure \ref{f1},
to which the estimated IWC profile is most likely to belong. In the second step, we estimate the IWC profile,
using a class specific ensemble Kalman Smoother (EKS) methodology similar to that of \cite{grecu2018}.  
The EKS algorithm updates the estimated IWC relative to the mean IWC of the class to which the profile belongs
based on the differences between the actual active and passive observations and their mean class values. The 
second step of this procedure is formally identical to the one used in \cite{grecu2018}, but the first step is
different. In \cite{grecu2018}, the first step was based on a simple distance-based evaluation. That strategy
is likely to be suboptimal in this study, because the joint distribution of IWC profiles and associated
observations are significantly more complex. We therefore use a more complex classification methodology
based on the TensorFlow library \citep{tensorflow2016}. The class estimation model is defined as a
TensorFlow Model with two dense layers of 30 neurons each, followed by a softmax layer \citep{deepL2016}. The 
class estimation model is trained using the 70\% of the simulated observations and the corresponding IWC 
profiles, the remaining 30\% of the data being used for evaluation. %The training is done using the Adam

\begin{equation}
    \mathbf{X}=\mathbf{\bar{X}_i}+\mathbf{Cov(X_i,Y_i)}\mathbf{Cov(Y_i,Y_i)}^{-1}(\mathbf{Y}-\mathbf{\bar{Y}_i}) \label{eq:enks}
\end{equation}
where $\mathbf{X}$ is the state variable describing the IWC profile, $\mathbf{Y}$ is the vector containing
the variation, $\mathbf{X_i}$ is the set of state variables for profiles in class $i$, and $\mathbf{Y_i}$ is the
set of associated observations.  Variables $\mathbf{\bar{X}_i}$ and $\mathbf{\bar{Y}_i}$ are the mean values
of the state variables and observations in class $i$, respectively. The covariance matrices between 
$\mathbf{X_i}$ and $\mathbf{Y_i}$ are denoted by $\mathbf{Cov(X_i,Y_i)}$. In step 1, the class is estimated
using the TensorFlow model, while in step 2, the IWC profile is estimated using the EKS algorithm summarized in 
Equation \ref{eq:enks}.

As already mentioned, a cross validation methodology is used for evaluation, with 70\% of 
the data used for training and the remaining 30\% of the data used for validation.  The partition of the data into 
training and 
evaluation subsets is done randomly.  Usually, the partition, training and evaluation steps are repeated 
several times.  However, given the fact that differences in the relationships between the ice property and 
their associated simulated observations are functions of the meteorological context, and that all regimes are 
well-sampled in both the training and testing subsets (e.g. out of every 10 pixels in a scene, about 7 end-up 
in the training dataset, while the others in the testing dataset), the repetition of the partition, training, 
and evaluation steps multiple times is not necessary. Therefore, in our evaluation, we partition the data into 
training and evaluation only once and perform all the evaluation for a single partition. The evaluation 
criteria include the correlation coefficient, the bias, and visual inspections of graphical representations of 
the estimated properties relative to their references.
%vs

\section{Results}
\subsection{Radiometer-only retrievals}
As previously mentioned, submillimeter-wave radiometers are likely to provide by themselves more complete 
information about the total amount of ice in their observing volumes than lidars or Ku-band radars with 
limited sensitivity. However, radiometers observations are an integrated measure of radiative process in 
clouds that provide little information about the vertical distribution of ice. From this perspective, an 
evaluation in terms of the ice water path ($IWP$) defined as the vertical integral of the $IWC$, i.e. 
$IWP=\int_0^{Z_{top}}IWC(z)dz$ is insightful.  Shown in Figure \ref{f4} is the frequency of IWP 
estimated from radiometer-only observations as a function of its true value. As apparent in the figure, there is good 
correlation between the retrieved and the true IWP values. The numerical value of the correlation coefficient 
is 0.92, and there is no-overall bias. That is, the mean values of retrieved $IWP$ and true $IWP$ values are 
equal. However, conditional biases are apparent, with overestimation of $IWP$ for values smaller than 
100 g/$m^2$ and some underestimation for values larger than 1000 g/$m^2$.  The biases at the low end of the 
$IWP$ range are not surprising, given that the impact caused by ice scattering on the total radiometric signal 
is small for low values of $IWP$ and hard to distinguish from other sources of variability in radiometer 
observations. Saturation effects are most likely responsible for underestimation at the high end.  It should 
be noted that in this evaluation, only atmospheric profiles that exhibit ice detectable by the CS radar are used. Therefore,
a radiometer-only estimation procedure derived from this training dataset is likely to result in significant 
overestimation if not used in conjunction with a discrimination procedure. However, such procedure is not critical 
in this study, as the lidar observations may be used to discriminate between clear skies and ice clouds.
\begin{figure}[t]
    \centering
    \includegraphics[width=0.75\textwidth,angle=0]{./Figs/fig04.png}\\
    \caption{Frequency plot of estimated IWP derived radiometer-observations as a function of the 
    true IWP used in observations synthesis}\label{f4}
\end{figure}
Although the radiometer-only estimation procedure is able to estimate the integrated amount of ice in clouds fairly
wells, its ability to characterize the vertical distribution of ice in clouds is limited.  Figure \ref{f5} shows the
conditional vertical distributions of the estimated and true IWC for the 18 classes described in Section 2a 
and shown in Figure \ref{f1}. As apparent in the figure, there are significant differences between the
estimated and true IWC profiles. 
\begin{figure}[t]
    \centering
    \includegraphics[width=0.75\textwidth,angle=0]{./Figs/fig05.png}\\
    \caption{True and radiometer-only retrieved conditional mean IWC for the 18 classes described in 
    Figure \ref{f1}.}\label{f5}
\end{figure}

Further insight into the radiometer-only estimation performance may be derived by defining the ice profile gravity center 
(GC) as $z_{GC}=\frac {\int_0^{Z_{top}}zIWC(z)dz} {\int_0^{Z_{top}}IWC(z)dz}$, where $z$ is the distance relative to 
the freezing level, the $Z_{top}$ is the distance from the top of the atmosphere to the freezing level.  
Shown in Figure \ref{f6} is the frequency of IWC gravity center
estimated from radiometer-only observations as a function of its true value.  It may be observed in the figure that while 
the true IWC gravity center exhibits quite a broad distribution, the one retrieved from the radiometer-only 
observations exhibits a multimodal narrow distribution. Moreover, there is almost no correlation between the retrieved 
and the true IWC gravity center.  This is another indication that, while the total amount of ice may be reasonably 
estimated from radiometer-only observations, its vertical distribution can not be determined from 
radiometer-only observations.

\begin{figure}[t]
    \centering
    \includegraphics[width=0.75\textwidth,angle=0]{./Figs/fig06.png}\\
    \caption{Same as in Figure 3, but for the $IWC$ gravity center.}\label{f6}
\end{figure}

\subsection{Synergistic retrievals}

From the synergy of the instrument perspective, it is useful to investigate how information easily derivable 
from the active instrument observation can complement the submillimeter-wave radiometer observations. The top of 
the clouds derived from the lidar backscatter observations is such information. To incorporate it into the $IWC$ 
retrievals, we simply extend the dimension of the input by one entry that contains the lidar-based cloud top 
estimate. This additional piece of information makes the retrievals more specific by confining the set of 
potential matches to those that are consistent from the cloud top perspective with the lidar observations.

Shown in Figure 5 is frequency plot of the $IWC$ gravity centers estimated from lidar and radiometer 
observations as a function of the true $IWC$ gravity centers. One may note in the figure a significant 
improvement relative to Figure 4.

\begin{figure}[t]
    \centering
    \includegraphics[width=0.75\textwidth,angle=0]{./Figs/fig07.png}\\
    \caption{.}\label{f7}
\end{figure}

\begin{figure}[t]
    \centering
    \includegraphics[width=0.75\textwidth,angle=0]{./Figs/fig08.png}\\
    \caption{.}\label{f8}
\end{figure}

\begin{figure}[t]
    \centering
    \includegraphics[width=0.75\textwidth,angle=0]{./Figs/fig09.png}\\
    \caption{.}\label{f9}
\end{figure}

\begin{table}[t]
\caption{Performance summary.}\label{t1}
\begin{center}
\begin{tabular}{ccccc}
\hline\hline
Instruments\\Score & Radiometer & Radar-Radiometer & Lidar-Radiometer & Radar-Lidar-Radiometer\\
\hline
NRMS & 0.73 &  0.59  &  0.32 & 0.22 \\
Class. Accurracy & 0.39 & 0.48 & 0.92 & 0.94 \\
\hline
\end{tabular}
\end{center}
\end{table}

% text...
% 
% \section{Section title}

%%%
% \section{First primary heading}

% \subsection{First secondary heading}

% \subsubsection{First tertiary heading}

% \paragraph{First quaternary heading}

%%%%%%%%%%%%%%%%%%%%%%%%%%%%%%%%%%%%%%%%%%%%%%%%%%%%%%%%%%%%%%%%%%%%%
% TABLES---INSERT NEAR IN-TEXT DISCUSSION
%%%%%%%%%%%%%%%%%%%%%%%%%%%%%%%%%%%%%%%%%%%%%%%%%%%%%%%%%%%%%%%%%%%%%
%%  Enter tables near where they are discussed within the document. 
%%  Please place tables before/after paragraphs, not within a paragraph.
%%
%


%%%%%%%%%%%%%%%%%%%%%%%%%%%%%%%%%%%%%%%%%%%%%%%%%%%%%%%%%%%%%%%%%%%%%
% FIGURES---INSERT NEAR IN-TEXT DISCUSSION
%%%%%%%%%%%%%%%%%%%%%%%%%%%%%%%%%%%%%%%%%%%%%%%%%%%%%%%%%%%%%%%%%%%%%
%%  Enter figures near where they are discussed within the document.
%%  Please place figures before/after paragraphs, not within a paragraph.
% %
%
%\begin{figure}[t]
%  \noindent\includegraphics[width=19pc,angle=0]{figure01.pdf}\\
%  \caption{Enter the caption for your figure here.  Repeat as
%  necessary for each of your figures. Figure from \protect\cite{Knutti2008}.}\label{f1}
%\end{figure}

\clearpage
%%%%%%%%%%%%%%%%%%%%%%%%%%%%%%%%%%%%%%%%%%%%%%%%%%%%%%%%%%%%%%%%%%%%%
% ACKNOWLEDGMENTS
%%%%%%%%%%%%%%%%%%%%%%%%%%%%%%%%%%%%%%%%%%%%%%%%%%%%%%%%%%%%%%%%%%%%%
\acknowledgments
%  Keep acknowledgments (note correct spelling: no ``e'' between the ``g'' and
% ``m'') as brief as possible. In general, acknowledge only direct help in
%  writing or research. Financial support (e.g., grant numbers) for the work done, 
%  for an author, or for the laboratory where the work was performed must be 
%  acknowledged here rather than as footnotes to the title or to an author's name.
%  Contribution numbers (if the work has been published by the author's institution 
%  or organization) should be placed in the acknowledgments rather than as 
%  footnotes to the title or to an author's name.

%%%%%%%%%%%%%%%%%%%%%%%%%%%%%%%%%%%%%%%%%%%%%%%%%%%%%%%%%%%%%%%%%%%%%
% DATA AVAILABILITY STATEMENT
%%%%%%%%%%%%%%%%%%%%%%%%%%%%%%%%%%%%%%%%%%%%%%%%%%%%%%%%%%%%%%%%%%%%%
% 
%
\datastatement
%  The data availability statement is where authors should describe how the data underlying 
%  the findings within the article can be accessed and reused. Authors should attempt to 
%  provide unrestricted access to all data and materials underlying reported findings. 
%  If data access is restricted, authors must mention this in the statement. See
%  {http://www.ametsoc.org/PubsDataPolicy} for more info.

%%%%%%%%%%%%%%%%%%%%%%%%%%%%%%%%%%%%%%%%%%%%%%%%%%%%%%%%%%%%%%%%%%%%%
% APPENDIXES
%%%%%%%%%%%%%%%%%%%%%%%%%%%%%%%%%%%%%%%%%%%%%%%%%%%%%%%%%%%%%%%%%%%%%
%
%% If only one appendix, use

%\appendix

%% If more than one appendix, use \appendix[<letter>], e.g.,

%\appendix[A] 

%% Appendix title is necessary! For appendix title:

%\appendixtitle{Title of Appendix}

%%% Appendix section numbering (note, skip \section and begin with \subsection)
%
% \subsection{First primary heading}

% \subsubsection{First secondary heading}

% \paragraph{First tertiary heading}


%%%%%%%%%%%%%%%%%%%%%%%%%%%%%%%%%%%%%%%%%%%%%%%%%%%%%%%%%%%%%%%%%%%%%
% REFERENCES
%%%%%%%%%%%%%%%%%%%%%%%%%%%%%%%%%%%%%%%%%%%%%%%%%%%%%%%%%%%%%%%%%%%%%
% Make your BibTeX bibliography by using these commands:
\bibliographystyle{ametsocV6}
\bibliography{references}


\end{document}
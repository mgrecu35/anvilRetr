\documentclass{article}
\usepackage{fullpage}
% use package for figures
\usepackage{graphicx}
%\renewcommand{\baselinestretch}{2}
\author{Mircea Grecu (Morgan State University), David Bolvin (SSAI) and \\George Huffman (NASA/GSFC)}
\title{Final Progress Report Grant 80NSSC19K0674\\ 
"Improved detection and quantification of precipitation by the TRMM/GPM combined algorithm"}
\date{}
\begin{document}
\maketitle

%\tableofcontents
%80NSSC22K0598
\section{Project description}
Ground clutter is signal in the radar observations caused by the ground. It may cause uncertainties 
in the radar precipitation estimates by obscuring precipitation signal.  
Ground clutter induced uncertainties tend to be large for shallow or low freezing-level precipitation 
systems in the off-nadir regions of the radar swath. This is because such systems are characterized by
large variations in the precipitation intensity with range, and simple schemes to estimate the surface
precipitation from observations not affected by ground clutter are usually not adequate. In this project,
we developed a methodology to mitigate the negative impact of  ground clutter on the detection 
and quantification of surface precipitation from both TRMM and GPM radar observations.  
The methodology is based on the fact that ground clutter is a function of the radar incidence angle,
i.e. it is a minimum at nadir and drastically increases with the viewing angle.  From an instantaneous and 
climatological perspective, it is necessary to accurately estimate precipitation at all radar 
viewing angles. We, therefore, developed a methodology to mitigate ground clutter issues using a 
comprehensive database of GPM near-nadir radar observations, minimally contaminated by ground 
clutter, matched with coincident GMI observations.  For a given off-nadir radar reflectivity 
profile, the database is used to identify records characterized by similar surface conditions
and precipitation type to reconstruct the information obscured by the ground clutter in the 
off-nadir profile.

\section{Final Progress Report}
% include figure
Shown in the left panel of Figure \ref{fig:fig1} is a cross-section through an observed reflectivity field at 
Ku-band (left panel). As apparent in the figure, the reflectivity field is increasingly contaminated 
by ground clutter (which results in much stronger echo than that usually associated with precipitation)
with increasing viewing angle. In the absence of more complete information, one may assume that
the precipitation rate is constant with height, and set the precipitation rate at the 
surface equal to that at the lowest clutter-free bin. However, this assumption is generally 
not valid and tends to lead to significant biases in the surface precipitation estimates.
The left sub-panel of the right panel of Figure \ref{fig:fig1} shows the average stratiform 
precipitation profiles over oceans stratified by the freezing level bin. As apparent in the figure,
the precipitation rate increases with range bin. A representation of the average stratiform precipitation
profiles as a function of the range relative to the freezing level bin and normalized by the
freezing level rate is also shown in the 
right panel of Figure \ref{fig:fig1}. This representation removes most of the variability apparent in the
absolute range bin representation and makes the different freezing level bins almost indistinguishable.
\begin{figure}[h]
\centering
\begin{tabular}{cc}
\includegraphics[width=0.485\textwidth]{./clutterCrossSect.png}&
\includegraphics[width=0.5\textwidth]{./stratProfiles.png}
\end{tabular}
\caption{Cross-section through observed reflectivity field at Ku-band (left panel) and average
stratiform precipitation profiles over oceans (right panel).}
\label{fig:fig1}
\end{figure}
Given the well-defined and robust variation of the precipitation rate with range, a simple, but
effective, method to mitigate the biases in the surface precipitation estimates may be formulated
by simply scaling the normalized precipitation rate profile to match the precipitation rate at the
lowest clutter-free bin and use the resulting profile to fill-in the values at longer ranges. 
Mathematically, this may be expressed as
\begin{equation}
    PrecipRate(Z<ZCF)=PrecipRate(ZCF) NCM(Z)/NCM(ZCF) 
\label{eq:eq1}
\end{equation}
where $Z$ is the height, $ZCF$ is the height of the lowest clutter-free bin, and
$NCM(Z)$ is the normalized conditional precipitation profile.

Equation \ref{eq:eq1} is applied conditionally on the surface and precipitation type with
different normalized average precipitation profiles being used for different surface conditions
and precipitation types. Its global implementation and application to the GPM Ku-band radar 
observations resulted in:
\begin{itemize}
\item More precipitation at the surface over oceans. The impact is more pronounced at higher 
latitudes because the freezing level is lower and the precipitation rate increases with range
is more pronounced above the freezing level.
\item Less precipitation at the surface over land in the tropics and sub-tropics. This is because
the precipitation rate tends to decrease with range below the freezing level due to evaporative
processes.  This is also the case at mid-latitudes in the summer.
\item More precipitation at the surface over land in the mid-latitudes in the winter. This is because
the precipitation rate tends to increase with range due to the fact that the lowest-clutter free
bin is usually above the freezing level and there are no significant evaporative process 
to reduce the precipitation rate.

\end{itemize}
\begin{itemize}
\item The DPR does not miss all the light precipitation associated with a given point in the 
Tb-space.
\item An empirical algorithm may be derived from collocated Tbs and DPR retrievals.
\item The empirical algorithm is to be applied only when the DPR does not detect precipitation. 
When estimates are greater than 0, a decision is required.
\end{itemize}
\begin{figure}[h]
    \centering
    \begin{tabular}{cc}
    \includegraphics[width=0.46\textwidth]{./highLatCMB.png}&
    \includegraphics[width=0.45\textwidth]{./highLatCMB_NN.png}
    \end{tabular}
    \caption{}
    \label{fig:fig1}
    \end{figure}

\end{document}